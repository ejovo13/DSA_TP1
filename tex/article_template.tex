
% This is a simple template for a LaTeX document using the "article" class.
% See "book", "report", "letter" for other types of document.

\documentclass[10pt]{article} % use larger type; default would be 10pt

\usepackage[T1]{fontenc}
\usepackage[french]{babel}
\usepackage[utf8]{inputenc} % set input encoding (not needed with XeLaTeX)
\usepackage{kpfonts}

%%% Examples of Article customizations
% These packages are optional, depending whether you want the features they provide.
% See the LaTeX Companion or other references for full information.

%%% PAGE DIMENSIONS
\usepackage{geometry} % to change the page dimensions
\geometry{a4paper} % or letterpaper (US) or a5paper or....
% THE MARGINS FOR DSA is 1.5cm
\geometry{margin=1.5cm} % for example, change the margins to 2 inches all round
% \geometry{landscape} % set up the page for landscape
%   read geometry.pdf for detailed page layout information


% \usepackage[parfill]{parskip} % Activate to begin paragraphs with an empty line rather than an indent

%%% PACKAGES
\usepackage{booktabs} % for much better looking tables
\usepackage{array} % for better arrays (eg matrices) in maths
\usepackage{paralist} % very flexible & customisable lists (eg. enumerate/itemize, etc.)
\usepackage{verbatim} % adds environment for commenting out blocks of text & for better verbatim
% \usepackage{subfig} % make it possible to include more than one captioned figure/table in a single float
% These packages are all incorporated in the memoir class to one degree or another...

%%% HEADERS & FOOTERS
\usepackage{fancyhdr} % This should be set AFTER setting up the page geometry
% \pagestyle{fancy} % options: empty , plain , fancy

\usepackage{graphicx} % support the \includegraphics command and options
\usepackage{subcaption}
\usepackage{caption}

\usepackage{dingbat} % For the pointy hands
\usepackage{pifont}
% \usepackage{xcolor} % For pretty colors
\usepackage[table]{xcolor}
\usepackage{tikz} % for nice pictures
\usepackage{blindtext}
\usepackage{wrapfig}
\usepackage{gensymb}
% \usepackage{table}

% COLORs
\definecolor{mygold}{RGB}{182, 153, 45}
\definecolor{mygreen}{RGB}{62, 171, 0}
\definecolor{dullgreen}{RGB}{165, 181, 45}
\definecolor{mypurp}{RGB}{84, 45, 181}


% \renewcommand{\headrule}{\color{gray}}
\renewcommand{\headrule}{\hbox to\headwidth{%
  \color{gray}\leaders\hrule height \headrulewidth\hfill}}

\renewcommand{\footrulewidth}{1pt}
% \renewcommand{\footrule}{\hbox to\headwirth{
%     \color{gray}\leaders\hrule height \footrulewidth\hfill}}

\renewcommand{\footrule}{{\color{gray}\vskip-\footruleskip\vskip-\footrulewidth \hrule width\headwidth height\footrulewidth\vskip\footruleskip}}

\fancyhf{}
\rhead{\textcolor{gray}{Séance TP 1}}
\chead{\color{gray} Travaux Pratiques}
% \lhead{Optimisation du GCC}
\lfoot{\color{gray}\textcopyright 2022 Evan Voyles}
\rfoot{\color{gray} Page \thepage\ sur 3}
\cfoot{\color{gray}Spécialité MAIN-3}
\footskip = 0pt
% \voffset = 10pt
% \headsep = 0pt
% \cfoot{\thepage\ of \pageref{LastPage}}

% \renewcommand{\headrulewidth}{0pt} % customise the layout...
% \lhead{}\chead{}\rhead{}
% \lfoot{}\cfoot{\thepage}\rfoot{}


\usepackage[absolute,overlay]{textpos} % Add text in any arbitrary position

%%% SECTION TITLE APPEARANCE
\usepackage{sectsty}
\allsectionsfont{\sffamily\mdseries\upshape} % (See the fntguide.pdf for font help)
% (This matches ConTeXt defaults)

%%% ToC (table of contents) APPEARANCE
\usepackage[nottoc,notlof,notlot]{tocbibind} % Put the bibliography in the ToC
\usepackage[titles,subfigure]{tocloft} % Alter the style of the Table of Contents
\renewcommand{\cftsecfont}{\rmfamily\mdseries\upshape}
\renewcommand{\cftsecpagefont}{\rmfamily\mdseries\upshape} % No bold!

%%% END Article customizations

%%% The "real" document content comes below...

\newcommand{\asgold}[1]{\textcolor{mygold}{{\bf#1}}}
\newcommand{\asgrey}[1]{\textcolor{gray}{{\bf#1}}}
\newcommand{\asred}[1]{\textcolor{red}{{\bf#1}}}
\newcommand{\asor}[1]{\textcolor{orange}{{\bf#1}}}
\newcommand{\ascy}[1]{\textcolor{cyan}{{\bf#1}}}
\newcommand{\asgr}[1]{\textcolor{mygreen}{{\bf#1}}}
\newcommand{\aspurp}[1]{\textcolor{mypurp}{{\bf#1}}}


\begin{document}

% \vspace{-10pt}
% This is the PERFECT SCALING, now we need to adjust the scale a little bit.
\begin{tikzpicture}[remember picture,overlay,yshift=-.2cm, xshift=1.75cm] % LMAOOOOOO This is the EXACT POSITION!!!!
    \node at (0,0) {\includegraphics[width=4.0cm,height=1.6cm]{media/1280px-Logo_Polytech_Sorbonne.png}};
\end{tikzpicture}

% \hspace{-5cm}
% \hspace{-1cm}
% \hspace{-1cm}

\vspace{-1cm}
\vspace{0.3cm}

{\raggedleft \color{mygold} Algorithmique générale\\
MAIN-3, année 2022\\
Séance TP N\degree 1\\
Février 2022\\
VOYLES Evan\\}

% This vspace accomodates my name
\vspace{-0.42cm}
\vspace{1.23cm}

{\Large \noindent \color{mygold} Objectif}

{\color{mygold}\noindent\rule{\textwidth}{1pt}}
\vspace{0cm}
\begin{itemize}
    \item[{\color{mygold}\ding{43}}] Complexité en moyenne.
\end{itemize}

\vspace{1.2cm}
{\color{dullgreen}\noindent \Large \bf Problème}

\vspace{-0.28cm}
{\vspace{0cm}\color{dullgreen}\noindent\rule{\textwidth}{1pt}}

\begin{textblock*}{10cm}(9.9cm,7.2cm) % {block width} (coords)
    \Large \aspurp{[Complexité en moyenne du tri rapide]}
\end{textblock*}

\begin{enumerate}
    % \item Réaliser un suivi à la trace de la procédure \verb|partitionBis| appliquée aux instances suivantes :
    \item Réaliser un suivi à la trace de la procédure \texttt{partitionBis} appliquée aux instances suivantes :
    \begin{equation*}
        I_1 = [2,6,0,4,3,1,5], I_2 = [7,6,5,4,3,2,1]\ \ \mathrm{ et }\ \ I_3 = [1,2,3,4,5,6,7].
    \end{equation*}
\end{enumerate}

% \newpage


% Test out a table!!!
\begin{table}[h!]
    \begin{tabular}{ll|ccccc}
    \hline
    Instruction                    & Description/Remarque                                          & T                         & indpiv & pospiv & x & j \\
    \hline
    partitionBis(I$_1$, 0, 5)      & \asgold{T \textless{}- I$_1$},  deb \textless{}- 0, fin \textless{}- 5 & \asgold{{[}2, 6, 0, 4, 3, 1, 5{]}} &        &        &   &   \\
    indpiv \textless{}- deb        &       \asgold{indipiv \textless{}- 0}                                                       & {[}2, 6, 0, 4, 3, 1, 5{]} & \asgold{0}      &        &   &   \\
    pospiv \textless{}- deb      &           \asgold{pospiv \textless{}- 0}                                                    & {[}2, 6, 0, 4, 3, 1, 5{]} & 0      & \asgold{0}      &   &   \\
    x \textless{}- T(deb)            &        \asgold{x \textless{}- T(0)}                                                       & {[}2, 6, 0, 4, 3, 1, 5{]} & 0      & 0      & \asgold{2} &   \\
    \asgr{[Pour]} j \textless{}- deb + 1  & \asgold{j \textless{}- 0 + 1}                                                              & {[}2, 6, 0, 4, 3, 1, 5{]} & 0      & 0      & 2 & \asgold{1} \\
    \asgr{[Pour]} j \textless{}= fin      & \aspurp{1 \textless{} 5} est vrai, la boucle continue                    & {[}2, 6, 0, 4, 3, 1, 5{]} & 0      & 0      & 2 & \aspurp{1} \\
    \asgr{\ \ \ [Si]\ \ } T(j) \textless{}= x       & \cellcolor{mypurp} 6 \textless{}= 2 est faux, on retourne a la boucle            & {[}\aspurp{2}, \aspurp{6}, 0, 4, 3, 1, 5{]} & 0      & 0      & 2 & 1 \\
    \asgr{[Pour]} j \textless{}- j + 1   &  \asgold{j \textless{} - 1 + 1}                            & {[}2, 6, 0, 4, 3, 1, 5{]} & 0      & 0      & 2 & \asgold{2} \\
    \asgr{[Pour]} j \textless{}= fin      & 2 \textless{}= 5 est vrai, la boucle continue                 & {[}2, 6, 0, 4, 3, 1, 5{]} & 0      & 0      & 2 & 2 \\
    \asgr{\ \ \ [Si]\ \ } T(j) \textless{}= x       & \cellcolor{mypurp} 0 \textless{}= 2 est vrai                                     & {[}\aspurp{2}, 6, \aspurp{0}, 4, 3, 1, 5{]} & 0      & 0      & 2 & 2 \\
    pospiv \textless{}- pospiv + 1 &         \asgold{pospiv \textless{}- 0 + 1}                                                      & {[}2, 6, 0, 4, 3, 1, 5{]} & 0      & \asgold{1}      & 2 & 2 \\
    \asgr{\ \ \ [Si]\ \ } j \textgreater pospiv     & 2 \textgreater{} 1 est vrai                                     & {[}2, 6, 0, 4, 3, 1, 5{]} & 0      & 1      & 2 & 2 \\
    echanger(T, pospiv, j)         &  \cellcolor{mygold}T(1) \textless{}- 0, T(2) \textless{}- 6                                         & {[}2, \asgold{0}, \asgold{6}, 4, 3, 1, 5{]} & 0      & 1      & 2 & 2 \\
    \asgr{[Pour]} j \textless{}- j + 1   &  \asgold{j \textless{} - 2 + 1}                             & {[}2, 0, 6, 4, 3, 1, 5{]} & 0      & 1      & 2 & \asgold{3} \\
    \asgr{[Pour]} j \textless{}= fin      & 3 \textless{}= 5 est vrai, la boucle continue                 & {[}2, 0, 6, 4, 3, 1, 5{]} & 0      & 1      & 2 & 3 \\
    \asgr{\ \ \ [Si]\ \ } T(j) \textless{}= x       & \cellcolor{mypurp} 4 \textless{}= 2 est faux                                     & {[}\aspurp{2}, 0, 6, \aspurp{4}, 3, 1, 5{]} & 0      & 1      & 2 & 3 \\
    \asgr{[Pour]} j \textless{}- j + 1   &  \asgold{j \textless{} - 3 + 1}                             & {[}2, 0, 6, 4, 3, 1, 5{]} & 0      & 1      & 2 & \asgold{4} \\
    \asgr{[Pour]} j \textless{}= fin      & 4 \textless{}= 5 est vrai, la boucle continue                 & {[}2, 0, 6, 4, 3, 1, 5{]} & 0      & 1      & 2 & 4 \\
    \asgr{\ \ \ [Si]\ \ } T(j) \textless{}= x       & \cellcolor{mypurp} 3 \textless{}= 2 est faux                                     & {[}\aspurp{2}, 0, 6, 4, \aspurp{3}, 1, 5{]} & 0      & 1      & 2 & 4 \\
    \asgr{[Pour]} j \textless{}- j + 1   &  \asgold{j \textless{} - 4 + 1}                             & {[}2, 0, 6, 4, 3, 1, 5{]} & 0      & 1      & 2 & \asgold{5} \\
    \asgr{[Pour]} j \textless{}= fin      & 5 \textless{}= 5 est vrai, la boucle continue                 & {[}2, 0, 6, 4, 3, 1, 5{]} & 0      & 1      & 2 & 5 \\
    \asgr{\ \ \ [Si]\ \ } T(j) \textless{}= x       & \cellcolor{mypurp} 1 \textless{}= 2 est vrai                                     & {[}\aspurp{2}, 0, 6, 4, 3, \aspurp{1}, 5{]} & 0      & 1      & 2 & 5 \\
    pospiv \textless{}- pospiv + 1 &      \asgold{pospiv \textless{}- 1 + 1}                                                         & {[}2, 0, 6, 4, 3, 1, 5{]} & 0      & \asgold{2}      & 2 & 5 \\
    \asgr{\ \ \ [Si]\ \ } j \textgreater pospiv     & 5 \textgreater{} 2 est vrai                                     & {[}2, 0, 6, 4, 3, 1, 5{]} & 0      & 2      & 2 & 5 \\
    echanger(T, pospiv, j)         & \cellcolor{mygold}T(2) \textless{}- 1, T(5) \textless{}- 6                                         & {[}2, 0, \asgold{1}, 4, 3, \asgold{6}, 5{]} & 0      & 2      & 2 & 5 \\
    \asgr{[Pour]} j \textless{}- j + 1   &  \asgold{j \textless{} - 5 + 1}                             & {[}2, 0, 1, 4, 3, 6, 5{]} & 0      & 2      & 2 & \asgold{6} \\
    \asgr{[Pour]} j \textless{}= fin      & 6 \textless{}= 5 est faux, on sort de la boucle               & {[}2, 0, 1, 4, 3, 6, 5{]} & 0      & 2      & 2 & 6 \\
    \asgr{\ \ \ [Si]\ \ } indpiv \textless{} pospiv   & 0 \textless{} 2 est vrai                                        & {[}2, 0, 1, 4, 3, 6, 5{]} & 0      & 2      & 2 & 6 \\
    echanger(T, indpiv, pospiv)    &  \cellcolor{mygold}T(0) \textless{}- 1, T(2) \textless{}- 2                                        & {[}\asgold{1}, 0, \asgold{2}, 4, 3, 6, 5{]} & 0      & 2      & 2 & 6 \\
    \hline
    \end{tabular}
    \end{table}

\vspace{-0.4cm}
\noindent Nombre d'\asgold{échanges} : 3, Nombre de \aspurp{comparaison} : 5

\vspace{0.2cm}
Notre tableau en sorti $[1, 0, \asred{2}, 4, 3, 6, 4]$ a été partitionné en deux parties autour du pivot $\asred{2}$. A gauche, l'on a $T_g = [1, 0]$ dont tous
les éléments $g \in T_g$ satisfaitent $g < \asred{2}$ et à droite l'on a $T_d = [4, 3, 6, 5] \mid \forall d \in T_d,\ d > \asred{2}$. La schéma de partition de
l'algorithme \texttt{partitionBis} parcours le tableau pour compter le nombre des valeurs qui sont inférieures au pivot. En le comptant, s'il il y a des
valuers plus grand que le pivot dans le soustableau à gauche, \texttt{partitionBis} va effectuer un échange. Donc, on verra que ce schéma doit opérer $n - 2$ comparaisons
des éléments du tableau à trier.

\newpage

\pagestyle{fancy}

% Test out a table!!!
\begin{table}[h!]
    \begin{tabular}{ll|ccccc}
    \hline
    % \rowcolor{red} Instruction                    & Description/Remarque                                          & T                         & indpiv & pospiv & x & j \\
    % \rowcolor{mygold}
    Instruction                    & Description/Remarque                                          & T                         & indpiv & pospiv & x & j \\
    \hline
    partitionBis(I$_2$, 0, 5)    & \asgold{T \textless{}- I$_2$},  deb \textless{}- 0, fin \textless{}- 5 & \asgold{{[}7, 6, 5, 4, 3, 2, 1{]}} &        &        &   &   \\
    indpiv \textless{}- deb      & \asgold{indpiv \textless{}- 0}                                                            & {[}7, 6, 5, 4, 3, 2, 1{]} & \asgold{0}      &        &   &   \\
    pospiv \textless{}- deb      & \asgold{pospiv \textless{}- 0}                                                             & {[}7, 6, 5, 4, 3, 2, 1{]} & 0      & \asgold{0}      &   &   \\
    x \textless{}- T(deb)            &        \asgold{x \textless{}- 7}                             & {[}7, 6, 5, 4, 3, 2, 1{]} & 0      & 0      & \asgold{7} &   \\
    \asgr{[Pour]} j \textless{}- deb + 1  &  \asgold{j \textless{}- 0 + 1}                                                              & {[}7, 6, 5, 4, 3, 2, 1{]} & 0      & 0      & 7 & \asgold{1} \\
    \asgr{[Pour]} j \textless{}= fin      & \aspurp{1 \textless{}= 5} est vrai, la boucle continue                 & {[}7, 6, 5, 4, 3, 2, 1{]} & 0      & 0      & 7 & \aspurp{1} \\
    \asgr{\ \ \ [Si]\ \ } T(j) \textless{}= x       & \cellcolor{mypurp} 6 \textless{}= 7 est vrai                                     & {[}\aspurp{7}, \aspurp{6}, 5, 4, 3, 2, 1{]} & 0      & 0      & 7 & 1 \\
    pospiv \textless{}- pospiv + 1 & \asgold{pospiv \textless{}- 0 + 1}                                            & {[}7, 6, 5, 4, 3, 2, 1{]} & 0      & \asgold{1}      & 7 & 1 \\
    \asgr{\ \ \ [Si]\ \ } j \textgreater{} pospiv     & \aspurp{1 \textgreater{} 1} est faux                                     & {[}7, 6, 5, 4, 3, 2, 1{]} & 0      & \aspurp{1}      & 7 & \aspurp{1} \\
    \asgr{[Pour]} j \textless{}- j + 1  &  \asgold{j \textless{}- 1 + 1}                                                              & {[}7, 6, 5, 4, 3, 2, 1{]} & 0      & 1      & 7 & \asgold{2} \\
    \asgr{[Pour]} j \textless{}= fin      & \aspurp{2 \textless{}= 5} est vrai, la boucle continue                 & {[}7, 6, 5, 4, 3, 2, 1{]} & 0      & 1      & 7 & \aspurp{2} \\
    \asgr{\ \ \ [Si]\ \ } T(j) \textless{}= x       & \cellcolor{mypurp} 5 \textless{}= 7 est vrai                                     & {[}\aspurp{7}, 6, \aspurp{5}, 4, 3, 2, 1{]} & 0      & 1      & 7 & 2 \\
    pospiv \textless{}- pospiv + 1  & \asgold{pospiv \textless{}- 1 + 1}               & {[}7, 6, 5, 4, 3, 2, 1{]} & 0      & \asgold{2}      & 7 & 2 \\
    \asgr{\ \ \ [Si]\ \ } j \textgreater{} pospiv     & \aspurp{2 \textgreater{} 2} est faux                                     & {[}7, 6, 5, 4, 3, 2, 1{]} & 0      & \aspurp{2}      & 7 & \aspurp{2} \\
    \asgr{[Pour]} j \textless{}- j + 1  &        \asgold{j \textless{}- 2 + 1}                                                        & {[}7, 6, 5, 4, 3, 2, 1{]} & 0      & 2      & 7 & \asgold{3} \\
    \asgr{[Pour]} j \textless{}= fin      & \aspurp{3 \textless{}= 5} est vrai, la boucle continue                 & {[}7, 6, 5, 4, 3, 2, 1{]} & 0      & 2      & 7 & \aspurp{3} \\
    \asgr{\ \ \ [Si]\ \ } T(j) \textless{}= x       & \cellcolor{mypurp} 4 \textless{}= 7 est vrai                                     & {[}\aspurp{7}, 6, 5, \aspurp{4}, 3, 2, 1{]} & 0      & 2      & 7 & 3 \\
    pospiv \textless{}- pospiv + 1  & \asgold{pospiv \textless{}- 2 + 1}               & {[}7, 6, 5, 4, 3, 2, 1{]} & 0      & \asgold{3}      & 7 & 3 \\
    \asgr{\ \ \ [Si]\ \ } j \textgreater{} pospiv     & \aspurp{3 \textgreater{} 3} est faux                                     & {[}7, 6, 5, 4, 3, 2, 1{]} & 0      & \aspurp{3}      & 7 & \aspurp{3} \\
    \asgr{[Pour]} j \textless{}- j + 1  &          \asgold{j \textless{}- 3 + 1}                                                      & {[}7, 6, 5, 4, 3, 2, 1{]} & 0      & 3      & 7 & \asgold{4} \\
    \asgr{[Pour]} j \textless{}= fin      & \aspurp{4 \textless{}= 5} est vrai, la boucle continue                 & {[}7, 6, 5, 4, 3, 2, 1{]} & 0      & 3      & 7 & \aspurp{4} \\
    \asgr{\ \ \ [Si]\ \ } T(j) \textless{}= x       & \cellcolor{mypurp} 3 \textless{}= 7 est vrai                                     & {[}\aspurp{7}, 6, 5, 4, \aspurp{3}, 2, 1{]} & 0      & 3      & 7 & 4 \\
    pospiv \textless{}- pospiv + 1  & \asgold{pospiv \textless{}- 3 + 1}               & {[}7, 6, 5, 4, 3, 2, 1{]} & 0      & \asgold{4}      & 7 & 4 \\
    \asgr{\ \ \ [Si]\ \ } j \textgreater{} pospiv     & \aspurp{4 \textgreater{} 4} est faux                                     & {[}7, 6, 5, 4, 3, 2, 1{]} & 0      & \aspurp{4}      & 7 & \aspurp{4} \\
    \asgr{[Pour]} j \textless{}- j + 1  &             \asgold{j \textless{}- 4 + 1}                                                   & {[}7, 6, 5, 4, 3, 2, 1{]} & 0      & 4      & 7 & \asgold{5} \\
    \asgr{[Pour]} j \textless{}= fin      & \aspurp{5 \textless{}= 5} est vrai, la boucle continue                 & {[}7, 6, 5, 4, 3, 2, 1{]} & 0      & 4      & 7 & \aspurp{5} \\
    \asgr{\ \ \ [Si]\ \ } T(j) \textless{}= x       & \cellcolor{mypurp} 2 \textless{}= 7 est vrai                                     & {[}\aspurp{7}, 6, 5, 4, 3, \aspurp{2}, 1{]} & 0      & 4      & 7 & 5 \\
    pospiv \textless{}- pospiv + 1  & \asgold{pospiv \textless{}- 4 + 1}               & {[}7, 6, 5, 4, 3, 2, 1{]} & 0      & \asgold{5}      & 7 & 5 \\
    \asgr{\ \ \ [Si]\ \ } j \textgreater{} pospiv     & \aspurp{5 \textgreater{} 5} est faux                                     & {[}7, 6, 5, 4, 3, 2, 1{]} & 0      & \aspurp{5}      & 7 & \aspurp{5} \\
    \asgr{[Pour]} j \textless{}- j + 1  &           \asgold{j \textless{}- 5 + 1}                                                     & {[}7, 6, 5, 4, 3, 2, 1{]} & 0      & 5      & 7 & \asgold{6} \\
    \asgr{[Pour]} j \textless{}= fin      & \aspurp{6 \textless{}= 5} est faux, la boucle termine                   & {[}7, 6, 5, 4, 3, 2, 1{]} & 0      & 5      & 7 & \aspurp{6} \\
    \asgr{\ \ \ [Si]\ \ } indpiv \textless{} pospiv   & \aspurp{0 \textless{} 5} est vrai                                        &  {[}7, 6, 5, 4, 3, 2, 1{]} & \aspurp{0}      & \aspurp{5}      & 7 & 6 \\
    echanger(T, indpiv, pospiv)    & \cellcolor{mygold} T(0) \textless{}- 1, T(5) \textless{}- 7                                        &  {[}\asgold{1}, 6, 5, 4, 3, 2, \asgold{7}{]}  & 0      & 5      & 7 & 6 \\
    \hline
    \end{tabular}
    \end{table}

\vspace{-.4cm}
\noindent Nombre d'\asgold{échanges} : 1, Nombre de \aspurp{comparaison} : 5

\vspace{0.2cm}
Chaque élément du tableau qu'on à comparé avec le pivot en était inférieur. Cependant, on a réussi a partitionner le tableau avec un
seul échange dans le dernier étape. Avec le pivot $\asred{7}$, $T_g = [1, 6, 5, 4, 3, 2]$ et il vérifie $g < \asred{7}\ \forall g \in T_g$.

% \newpage

% Third table!!!!
\begin{table}[h!]
    \begin{tabular}{ll|ccccc}
    \hline
    % \rowcolor{red} Instruction                    & Description/Remarque                                          & T                         & indpiv & pospiv & x & j \\
    % \rowcolor{mygold}
    Instruction                    & Description/Remarque                                          & T                         & indpiv & pospiv & x & j \\
    \hline
    partitionBis(I$_3$, 0, 5)    & \asgold{T \textless{}- I$_3$},  deb \textless{}- 0, fin \textless{}- 5 & \asgold{{[}1, 2, 3, 4, 5, 6, 7{]}} &        &        &   &   \\
    indpiv \textless{}- deb      & \asgold{indpiv \textless{}- 0}                                                            & {[}1, 2, 3, 4, 5, 6, 7{]} & \asgold{0}      &        &   &   \\
    pospiv \textless{}- deb      & \asgold{pospiv \textless{}- 0}                                                             & {[}1, 2, 3, 4, 5, 6, 7{]} & 0      & \asgold{0}      &   &   \\
    x \textless{}- T(deb)            &        \asgold{x \textless{}- 1}                             & {[}1, 2, 3, 4, 5, 6, 7{]} & 0      & 0      & \asgold{1} &   \\
    \asgr{[Pour]} j \textless{}- deb + 1  &  \asgold{j \textless{}- 0 + 1}                                                              & {[}1, 2, 3, 4, 5, 6, 7{]} & 0      & 0      & 1& \asgold{1} \\
    \asgr{[Pour]} j \textless{}= fin      & \aspurp{1 \textless{}= 5} est vrai, la boucle continue                 & {[}1, 2, 3, 4, 5, 6, 7{]} & 0      & 0      & 1& \aspurp{1} \\
    \asgr{\ \ \ [Si]\ \ } T(j) \textless{}= x       & \cellcolor{mypurp} 2 \textless{}= 1 est faux & {[}\aspurp{1}, \aspurp{2}, 3, 4, 5, 6, 7{]} & 0      & 0      & 1& 1 \\
    \asgr{[Pour]} j \textless{}- j + 1  &  \asgold{j \textless{}- 1 + 1}                                                              & {[}1, 2, 3, 4, 5, 6, 7{]} & 0      & 0      & 1& \asgold{2} \\
    \asgr{[Pour]} j \textless{}= fin      & \aspurp{2 \textless{}= 5} est vrai, la boucle continue                 & {[}1, 2, 3, 4, 5, 6, 7{]} & 0      & 0      & 1& \aspurp{2} \\
    \asgr{\ \ \ [Si]\ \ } T(j) \textless{}= x       & \cellcolor{mypurp} 3 \textless{}= 1 est faux                                     & {[}\aspurp{1}, 2, \aspurp{3}, 4, 5, 6, 7{]} & 0      & 1      & 1& 2 \\

    \asgr{[Pour]} j \textless{}- j + 1  &  \asgold{j \textless{}- 2 + 1}                                                              & {[}1, 2, 3, 4, 5, 6, 7{]} & 0      & 0      & 1& \asgold{3} \\
    \asgr{[Pour]} j \textless{}= fin      & \aspurp{3 \textless{}= 5} est vrai, la boucle continue                 & {[}1, 2, 3, 4, 5, 6, 7{]} & 0      & 0      & 1& \aspurp{3} \\
    \asgr{\ \ \ [Si]\ \ } T(j) \textless{}= x       & \cellcolor{mypurp} 4 \textless{}= 1 est faux                                     & {[}\aspurp{1}, 2, 3, \aspurp{4}, 5, 6, 7{]} & 0      & 1      & 1& 3 \\
    \asgr{[Pour]} j \textless{}- j + 1  &  \asgold{j \textless{}- 3 + 1}                                                              & {[}1, 2, 3, 4, 5, 6, 7{]} & 0      & 0      & 1& \asgold{4} \\
    \asgr{[Pour]} j \textless{}= fin      & \aspurp{4 \textless{}= 5} est vrai, la boucle continue                 & {[}1, 2, 3, 4, 5, 6, 7{]} & 0      & 0      & 1& \aspurp{4} \\
    \asgr{\ \ \ [Si]\ \ } T(j) \textless{}= x       & \cellcolor{mypurp} 5 \textless{}= 1 est faux                                     & {[}\aspurp{1}, 2, 3, 4, \aspurp{5}, 6, 7{]} & 0      & 1      & 1& 4 \\
    \asgr{[Pour]} j \textless{}- j + 1  &  \asgold{j \textless{}- 4 + 1}                                                              & {[}1, 2, 3, 4, 5, 6, 7{]} & 0      & 0      & 1& \asgold{5} \\
    \asgr{[Pour]} j \textless{}= fin      & \aspurp{5 \textless{}= 5} est vrai, la boucle continue                 & {[}1, 2, 3, 4, 5, 6, 7{]} & 0      & 0      & 1& \aspurp{5} \\
    \asgr{\ \ \ [Si]\ \ } T(j) \textless{}= x       & \cellcolor{mypurp} 6 \textless{}= 1 est faux                                     & {[}\aspurp{1}, 2, 3, 4, 5, \aspurp{6}, 7{]} & 0      & 1      & 1& 5 \\
    \asgr{[Pour]} j \textless{}- j + 1  &  \asgold{j \textless{}- 5 + 1}                                                              & {[}1, 2, 3, 4, 5, 6, 7{]} & 0      & 0      & 1& \asgold{6} \\
    \asgr{[Pour]} j \textless{}= fin      & \aspurp{6 \textless{}= 5} est faux, la boucle termine                   & {[}1, 2, 3, 4, 5, 6, 7{]} & 0      & 0      & 1 & \aspurp{6} \\
    \asgr{\ \ \ [Si]\ \ } indpiv \textless{} pospiv   & \aspurp{0 \textless{} 0} est faux &  {[}1, 2, 3, 4, 5, 6, 7{]} & \aspurp{0}      & \aspurp{0}      & 1 & 6 \\
    \hline
    \end{tabular}
    \end{table}

\vspace{-.4cm}
\noindent Nombre d'\asgold{échanges} : 0, Nombre de \aspurp{comparaison} : 5
\newpage

% \vspace{0.2cm}
% Chaque élément du tableau qu'on à comparé avec le pivot en était inférieur. Cependant, on a réussi a partitionner le tableau avec un
% seul échange dans le dernier étape. Avec le pivot $\asred{7}$, $T_g = [1, 6, 5, 4, 3, 2]$ et il vérifie $g < \asred{7}\ \forall g \in T_g$.
Tout les éléments de $I_3 = [1, 2, 3, 4, 5, 6, 7]$ sont déjà triés donc on effectue aucun échange. On constate qu'il y a encore $n - 2 = 5$ comparaisons parce qu'on a
traversé le tableau à partir du premier élément jusqu'a l'avant-dernier élément.


\begin{itemize}
    \item[2.] Démontrer que la procédure \texttt{partitionBis} est correcte et analyser sa complexité.
\end{itemize}

\begin{itemize}
    \item[3.] Quels changements, s'ils existent, à apporter au pseudo-code du tri rapide?
\end{itemize}

\begin{itemize}
    \item[4.] Conduire une analyse de complexité en moyenne du tri rapide utilisant la procédure \texttt{partitionBis} à la
    place de la procédure \texttt{partition}.
\end{itemize}

\begin{itemize}
    \item[5.] D'après votre expérimentation, laquelle des deux méthodes \texttt{partition} et \texttt{partitionBis} est la
    plus efficace?
\end{itemize}

\end{document}
