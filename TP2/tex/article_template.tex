
% This is a simple template for a LaTeX document using the "article" class.
% See "book", "report", "letter" for other types of document.

\documentclass[10pt]{article} % use larger type; default would be 10pt

\input{preamble.tex}

\begin{document}

% \vspace{-10pt}
% This is the PERFECT SCALING, now we need to adjust the scale a little bit.
\begin{tikzpicture}[remember picture,overlay,yshift=-.2cm, xshift=1.75cm] % LMAOOOOOO This is the EXACT POSITION!!!!
    \node at (0,0) {\includegraphics[width=4.0cm,height=1.6cm]{media/1280px-Logo_Polytech_Sorbonne.png}};
\end{tikzpicture}

% \hspace{-5cm}
% \hspace{-1cm}
% \hspace{-1cm}

\vspace{-1cm}
\vspace{0.3cm}

{\raggedleft \color{mygold} Algorithmique générale\\
MAIN-3, année 2022\\
Séance TP N\degree 2\\
Février 2022\\
VOYLES Evan\\}

% This vspace accomodates my name
\vspace{-0.42cm}
\vspace{1.23cm}

{\Large \noindent \color{mygold} Objectif}

{\color{mygold}\noindent\rule{\textwidth}{1pt}}
\vspace{0cm}
\begin{itemize}
    \item[{\color{mygold}\ding{43}}] Problème dans des graphes.
\end{itemize}

\vspace{1.2cm}
{\color{dullgreen}\noindent \Large \bf Problème}

\vspace{-0.28cm}
{\vspace{0cm}\color{dullgreen}\noindent\rule{\textwidth}{1pt}}

\begin{textblock*}{10cm}(12.16cm,7.2cm) % {block width} (coords)
    \Large \aspurp{[Parcours dfs et applications]}
\end{textblock*}

\vspace{1cm}

La documentation générale pour ce travail se trouve \href{https://polytech-sorbonne-main-tp2.readthedocs.io/en/latest/}{ici}, la documentation
technique se trouve \href{https://ejovo13.github.io/DSA_TP1/}{là}, le répo \href{https://github.com/ejovo13/DSA_TP1}{git}.

\vspace{1cm}
\noindent \asgold{Partie A :} \aspurp{Préliminaires}

\begin{enumerate}
    % \item Réaliser un suivi à la trace de la procédure \verb|partitionBis| appliquée aux instances suivantes :
    \item Implémenter une procédure permettant d'importer un graphe à partir d'un fichier.
\end{enumerate}

La fonction \texttt{readGraph} qui prend en argument le nom d'un fichier

\begin{itemize}
    \item [2.] Implémenter une procédure permettant d'exporter au format \texttt{.dot} un graphe donné en argument.
\end{itemize}

\begin{itemize}
    \item [3.] Implémenter une procédure permettant de visualiser l'arbre dfs d'un graphe donné en entrée.
\end{itemize}

\begin{itemize}
    \item [4.] Implémenter une procédure testant si un graphe est connexe.
\end{itemize}

\begin{itemize}
    \item [5.] Implémenter une procédure inversant un graphe orienté donné en argument.
\end{itemize}

\vspace{.5cm}
\noindent \asgold{Partie B :} \aspurp{Composantes fortement connexes}

\begin{itemize}
    \item [1.] Implémenter un algorithme déterminant les compsantes fortement connexes d'un graphe orienté.
\end{itemize}

\begin{itemize}
    \item [2.] Analyser la complexité de votre algorithme.
\end{itemize}

\begin{itemize}
    \item [3.] Implémenter une procédure permettant d'exporter au format \texttt{dot} un graphe donné en argument et indiquant ses composantes fortement connexes.
\end{itemize}

\vspace{.5cm}
\noindent \asgold{Partie C :} \aspurp{Orientation forte d'un graphe}

\begin{itemize}
    \item [1.] Implémenter une procédure testant si un graphe est sans pont. Quelle est sa complexité?
\end{itemize}

\begin{itemize}
    \item [2.] Démontrer qu'un graphe $G$ est fortement orientable si et seulement si $G$ ne possède pas de pont.
\end{itemize}

\begin{itemize}
    \item [3.] Implémenter une procédure déterminant une orientation forte d'un graphe non orienté sans pont. Analyser sa complexité.
\end{itemize}

\begin{itemize}
    \item[4.] Exécecuter l'algorithme
\end{itemize}



% \newpage
\end{document}